\section{Motivation}


Deep learning is a family of data driven techniques that have been applied in various professional domains, such as medical diagnostics, autonomous driving, drug discovery and many more.
The purpose of these techniques is to help human actors make better decisions from observing a large amount of data.

On an abstract level, given input data $x$, deep learning models can be expressed as $f(x; w)$, where $f$ is a function of $x$ and $w$ refers to unknown parameters in the model, which are then inferred from data. This is why deep learning is called \emph{data driven}: a large proportion of the mathematical expression $f$ is inferred from and hence determined by data.

There are two main steps in deep learning: \emph{training} and \emph{inference}.
\emph{Training} is the process where the unknown parameters $w$ are being derived from a given set of data in an iterative manner and \emph{inference} is the application of the function $f$ given these inferred $w$. Because of the potentially large number of unknown parameters and the highly complex form of $f$, the training process can take a very long time to complete.

Data driven approaches can be very powerful. They apply patterns that go beyond human knowledge and experience. However, the downside is that these techniques are hard to verify, interpret and reproduce.

If we take one step back and further break down the construction of a deep learning program, it usually goes like this:
\begin{itemize}
\item {\bf Step 1}: construct $f(\cdot; w\mid h)$. This step is sometimes called the \emph{first selection problem} or \emph{model selection} in machine learning. There are three parts in constructing this function: 1) the general form of the mathematical expression $f$, 2) \emph{hyperparameters} $h$ and 3) the unknown parameters $w$. The unknown parameters will be inferred from data automatically at a later stage, but as the developer, one needs to first manually choose 1) and 2). The hyperparameters $h$ refer to a set of parameters that can be changed to modify the form of the function.
  Unlike the unknown parameters $w$, the hyperparameters are typically determined by human developers. For that reason, we write $h$ on the right side of the vertical bar $\mid$ to indicate that it is a set of \emph{given} values.
  There are two sub-steps in this process:
  \begin{itemize}
  \item {\bf Step 1.1}: choose $f$ and $h$ based on theories, experience and human intuition
  \item {\bf Step 1.2}: implement the deep learning model as a computer program
  \end{itemize}
\item {\bf Step 2}: derive the unknown parameters $w$ from a set of data $\mathcal{X}_0$. This step is referred to as the \emph{second selection problem} or \emph{training} in machine learning. For this reason, $w$ is usually called the \emph{trainable parameters}. This is an automatic process typically implemented by backpropagation. Here we use $\hat{w}$ to denote the estimates of the unknown parameters $w$.
\item {\bf Step 3}: evaluate the model $f(\cdot; \hat{w}\mid h)$ on a separate data set $\mathcal{X}_1$ by applying the function $f(x; \hat{w}\mid h), \forall x\in\mathcal{X}_1$ and compare the results to the ground truth. This step is to validate the choice of $f$ and $h$ by evaluating the quality of $\hat{w}$. Note that since $\hat{w}$ is derived from $\mathcal{X}_0$, the validation has to be based on a different set of data $\mathcal{X}_1$.
\item Iterate from step 1 until a desirable outcome is achieved by step 3.
\end{itemize}

\paragraph{What could possibly go wrong?}
At a first glance, the process seems well established. However, there are a couple of challenges when it comes to deep learning.
\begin{itemize}
\item Step 1.1: it is difficult to choose and keep track of $f$ and $h$ due to the following reasons.
\begin{itemize}
\item Infinitely large search space. Loosely speaking, $f$ can be seen as a composition of an arbitrary number of deep learning layer functions, such as two dimensional convolution functions, fully connected functions, batch normalization functions, etc. Obviously, there are infinitely many different possible combinations of these layer functions. There are some rules of thumb regarding how to make a reasonable choice for the problem at hand, but it is nearly impossible to make statements about their actual behaviors. When it comes to the choice of $h$, it is even more fuzzy. In practice, $h$ refers to, for instance, the number of filters in a convolutional layer function, and this number can be arbitrary. There are many of these choices to be made in each layer function, which makes the whole problem completely intractable.
\item Slow feedback loop. To make things worse, once a choice is made, the feedback of the evaluation is incredibly slow.
\item Hard to reason about the choice.
\item Poor reproducibility. When $w$ is derived, the network can be used for inference.
  The network is then represented as a graph. There are libraries and tools to store and document the trained network as a graph in a easily accessible way, but the details of the training process are often lost. There are checkpoint systems used in different frameworks, but in most cases, the checkpoint needs to be loaded together with the Python code. There is no guarantee that at the point of loading the checkpoint, the code has not been modified compared to its original version. For this reason, it is quite challenging to reproduce the complete configuration of a network.
\end{itemize}
\item Step 1.2: it is easy to make errors in the implementation
\begin{itemize}
\item Deep learning programs are generally hard to debug.
  \begin{itemize}
  \item The semantics of the deep learning functions are hard to analyze and verify. There are simply too many uncertainties and variables that might have an impact on the output of the network.
  \item The effect of changes in code is hard to measure and keep track. For example, what is the impact on the output if I change a kernel size from $3\times 3$ to $7\times 7$?
  \item Missing components are hard to spot. There are some optional function calls during a neural network construction, for example, the initialization of trainable parameters $w$.
    They are optional in the sense that if the initialization step is omitted, the program will simply use the default functions and values instead of crashing.
    However, a different choice at this step can make a big difference to the final value of $w$. This will have an impact on the performance of the network during the inference phase. If the user is not aware of the consequences, it would be very difficult to understand that this is the root cause of the unexpected performance.
\item Large complex functions with implicit states and side effects make the programs difficult to reason, which makes it also hard to understand the data flow and the composition logic.
    \item Accumulated legacy bugs and errors. Deep learning code takes a long time to develop due to its high complexity and long training time. This nature makes it attempting to reuse code and programs that are previously written by the user themselves or by others. Due to the fact that the programs are not easy to test and verify, users typically treat these reusable modules as a black box (as long as they seem to work) and build their modules on top of them. As a result, there are many bugs and errors accumulating over time until the functionality breaks. In practice, bug discovery is more of a community effort than a principled process.
  \end{itemize}
\end{itemize}
\end{itemize}
In summary, there are a lot of errors being made along the way. In this work, we aim to improve the situation by providing the user with a eDSL for early bug discovery and network analysis and some other stuff. Not done here.

In this paper, we describe a domain specific language with the following purposes:
\begin{itemize}
\item (This might not be too relevant here or maybe describe it in a different way?) Working towards a more declarative representation. Encode the network into a data structure so that it can be easily manipulated, documented for reproducibility and compared to each other.
\item (This might not be too relevant either?) Minimal dependency. The data structure used to represent the neural network is an abstract syntax tree of the DSL. This AST does not depend on any deep learning frameworks, which means we do not need to install and instantiate the frameworks in order to enable the analysis.
\item Enabling analysis, testing and verification. Describe how we annotate the tree and do stuff. When traversing the AST, we can generate code to define a neural network in any major frameworks, generate appropriate testing code, analyze different properties of the neural network. As a future step, the AST can also be used for verification purposes.
\end{itemize}






%%% Local Variables:
%%% mode: latex
%%% TeX-master: "paper"
%%% End:
