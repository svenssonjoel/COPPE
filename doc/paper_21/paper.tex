\documentclass[sigplan,screen,authordraft]{acmart}

%%
%% \BibTeX command to typeset BibTeX logo in the docs
\AtBeginDocument{%
  \providecommand\BibTeX{{%
    \normalfont B\kern-0.5em{\scshape i\kern-0.25em b}\kern-0.8em\TeX}}}

%% Rights management information.  This information is sent to you
%% when you complete the rights form.  These commands have SAMPLE
%% values in them; it is your responsibility as an author to replace
%% the commands and values with those provided to you when you
%% complete the rights form.

\setcopyright{acmcopyright}
\copyrightyear{2018}
\acmYear{2018}
\acmDOI{10.1145/1122445.1122456}

\acmConference[Woodstock '18]{Woodstock '18: ACM Symposium on Neural
  Gaze Detection}{June 03--05, 2018}{Woodstock, NY}
\acmBooktitle{Woodstock '18: ACM Symposium on Neural Gaze Detection,
  June 03--05, 2018, Woodstock, NY}
\acmPrice{15.00}
\acmISBN{978-1-4503-XXXX-X/18/06}




\begin{document}

%\title{To be decided}
\title{}

\author{Yinan}
\affiliation{%
  \institution{}
  \streetaddress{}
  \city{}
  \country{}}
\email{}
\author{Joel}
\affiliation{%
  \institution{}
  \streetaddress{}
  \city{}
  \country{}}
\email{}
\author{Mary}
\affiliation{%
  \institution{}
  \streetaddress{}
  \city{}
  \country{}}
\email{}

\renewcommand{\shortauthors}{Yu, et al.}

\begin{abstract}

  abstracty stuff
  
\end{abstract}

%%
%% The code below is generated by the tool at http://dl.acm.org/ccs.cfm.
%% Please copy and paste the code instead of the example below.
%%
\begin{CCSXML}
<ccs2012>
 <concept>
  <concept_id>10010520.10010553.10010562</concept_id>
  <concept_desc>Computer systems organization~Embedded systems</concept_desc>
  <concept_significance>500</concept_significance>
 </concept>
 <concept>
  <concept_id>10010520.10010575.10010755</concept_id>
  <concept_desc>Computer systems organization~Redundancy</concept_desc>
  <concept_significance>300</concept_significance>
 </concept>
 <concept>
  <concept_id>10010520.10010553.10010554</concept_id>
  <concept_desc>Computer systems organization~Robotics</concept_desc>
  <concept_significance>100</concept_significance>
 </concept>
 <concept>
  <concept_id>10003033.10003083.10003095</concept_id>
  <concept_desc>Networks~Network reliability</concept_desc>
  <concept_significance>100</concept_significance>
 </concept>
</ccs2012>
\end{CCSXML}

\ccsdesc[500]{Computer systems organization~Embedded systems}
\ccsdesc[300]{Computer systems organization~Redundancy}
\ccsdesc{Computer systems organization~Robotics}
\ccsdesc[100]{Networks~Network reliability}

\keywords{neural networks}

%% A "teaser" image appears between the author and affiliation
%% information and the body of the document, and typically spans the
%% page.
%% \begin{teaserfigure}
%%   \includegraphics[width=\textwidth]{sampleteaser}
%%   \caption{Seattle Mariners at Spring Training, 2010.}
%%   \Description{Enjoying the baseball game from the third-base
%%   seats. Ichiro Suzuki preparing to bat.}
%%   \label{fig:teaser}
%% \end{teaserfigure}

\maketitle


% PLAN: (May 14)
%
% Description of Coppe
%   - How we model deep learning
%   - Easy to analyse
%   - Yaml interfaces/EDSL interface (Monad, Arrow)
%
% EDSL Programming example.
%   - Digit recognition? (find tutorial, generate the same code)
%
% Generate python *
%   - It is not hard to argue that deep learning code is full of errors
%   - Show that our approach is less full of errors.
%   - Maybe proof of concept
%
% Case study
%   - Look at some 
%
% Evaluation
%   - What can we show?
%   - What kind of errors are users making
%     - Example error -> coppe helps avoid/correct it.
%
% Future work
%   - Speed / memory
%   - Database
%   - Vanishing gradient
%   - 

\section{Introduction}

Introductory stuff


\begin{acks}
  acknowledgy stuff
\end{acks}

\bibliographystyle{ACM-Reference-Format}
\bibliography{bib}

\end{document}
